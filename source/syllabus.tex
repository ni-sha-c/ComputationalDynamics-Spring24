\documentclass[12pt]{article}
\usepackage[tagged, highstructure]{accessibility}
\usepackage[english]{babel}
\usepackage[utf8x]{inputenc}
\usepackage[T1]{fontenc}
\usepackage[margin=1in]{geometry}
\usepackage{scribe}
\usepackage{listings}
\usepackage{natbib,verbatim}
\usepackage{hyperref}
\hypersetup{
    colorlinks=true,
    linkcolor=blue,
    filecolor=magenta,      
    urlcolor=magenta,
    pdftitle={Course Syllabus},
    pdfauthor={Nisha Chandramoorthy},
    pdflang={en-US}
}

%\Scribe{Your Name}
\title{Syllabus-CDS-CSE8803}
\Lecturer{Nisha Chandramoorthy (nishac@gatech.edu)}
\LectureNumber{Foundations of Computational Dynamical Systems}
\LectureDate{Jan 2, 2024}
\LectureTitle{}

\lstset{style=mystyle}

\begin{document}
\MakeScribeTop

Dynamical systems, often represented by ordinary differential equations (ODEs) or iterated function systems, are models for variables that evolve with time. Dynamical systems theory is the study of the structure of all possible time-varying solutions of such systems, called orbits or trajectories. When the dynamics is stochastic or deterministic but chaotic, it also makes sense to study the time-asymptotic behavior or \emph{ergodic} properties of orbits. Across science and engineering, several computational questions are asked of dynamical systems such as: i) how sensitive are the orbits and orbit structure of the dynamics to one-time perturbations, to persistent deterministic perturbations, and to noisy perturbations?
ii) given some measurements or observations of the dynamics, how do we forecast orbits into the future or predict their statistical moments?, and iii) how can we construct a simpler dynamical system (with fewer variables) that captures its \emph{essential} features? Putting such questions in a mathematical language and studying rigorous methods for their solutions are the primary goals of this course.

We will start with an overview of nonlinear deterministic systems and their numerical analysis, using dynamical systems and ergodic theory. We will then cover some stochastic analysis and methods from computational statistics. See section \ref{sec:topics} for a list of topics. After establishing mathematical foundations, we will discuss rigorous computations applicable to dynamical systems that arise in the geosciences and in machine learning (optimization algorithms).

\section{General information}
\begin{itemize}
	\item 3 credits, two lectures per week, 6 homeworks and 1 final project (no exams).
	\item Class time and location: Mondays and Wednesdays, 2:00 pm -- 3:15 pm, Molecular Sciences and Engineering building G021
	\item Class dates: Jan 08, 2024 -- May 02, 2024.
	\item Office hours: Fridays 10:00 am -- 11:00 am
	\item Instructor email: nishac@gatech.edu
\end{itemize}

\section{Prerequisites and tips for success}
A strong background in linear algebra, probability and statistics as well as mathematical maturity (proof-based math, e.g., real and complex analysis) are necessary to succeed in this course. You must be skilled at Python/Julia programming for ML/data science/scientific computing. Additionally, you must be motivated to gain a foundational understanding of dynamical systems and stochastic analysis and interested in their scientific applications. In this case, you can definitely fill in gaps in your math and computing background by working through textbook material in analysis, linear algebra and through programming assignments.

We emphasize that helping each other understand the concepts and enjoy the mathematical and computational aspects of learning together is the main goal of this course. Spending time building a strong foundation by learning classical techniques will help in forming a good mental model of this vast field, and help us keep up with (and not be intimidated by) the proliferating research in the area of data science methods.

\section{Resources (not exhaustive)}
\label{sec:resources}
Being interdisciplinary, this course will cover select content from multiple textbooks (and also research articles, which will be cited during class), some of which are listed below. 
\begin{itemize}
	\item 
		\href{https://books.google.com/books?hl=en&lr=&id=9nL7ZX8Djp4C&oi=fnd&pg=PR7&dq=katok+and+hasselblatt&ots=oWieU2cCAH&sig=T-g4-msNhCCTTpoSyMb09ZhnF2Y#v=onepage&q=katok%20and%20hasselblatt&f=false}{Introduction to the modern theory of dynamical systems} by Katok and Hasselblatt, Cambridge University Press.
	\item  \href{https://books.google.com/books?hl=en&lr=&id=wUBvDwAAQBAJ&oi=fnd&pg=PT7&ots=AOtaTsiIyX&sig=MexxyAiFbXhsxZNKq6V7YLjjv5w#v=onepage&q&f=false}{Nonlinear dynamics and chaos} by Strogatz, second edition, Westview Press.
	\item \href{https://link.springer.com/book/10.1007/978-3-662-12616-5}{Numerical Solution of Stochastic Differential Equations} by Kloeden and Platen, Springer. 

	\item \href{https://www.nowpublishers.com/article/Details/MAL-073}{Computational Optimal Transport} by Peyre and Cuturi, Foundations and Trends in Machine Learning, 2019.
	\item \href{https://books.google.com/books?hl=en&lr=&id=vaFKLXvfSaUC&oi=fnd&pg=PR9&dq=henk+dijkstra&ots=DCHHbStuIf&sig=ibiNU5CqYdT-Fo37v1FUIpuLs1o#v=onepage&q=henk%20dijkstra&f=false}{Nonlinear Physical Oceanography} by Dijkstra, Springer.	
	\item \href{https://books.google.com/books?hl=en&lr=&id=dWB9DwAAQBAJ&oi=fnd&pg=PR5&dq=foundations+of+machine+learning+mohri&ots=AznTXOq_s4&sig=oFBecq2rS2nusMY-xRj1qD-0Dsk#v=onepage&q=foundations%20of%20machine%20learning%20mohri&f=false}{Foundations of machine learning} by Mohri, Rostamizadeh and Talwalkar, second edition, MIT Press. 
	
\end{itemize}
Lecture notes and handouts will be posted on the \href{https://github.com/ni-sha-c/ComputationalDynamics-Spring24}{Github site} for the class.

\section{List of topics}
\label{sec:topics}
Please note that the plan below is subject to change, both in terms of the content and order. We will introduce mathematical analyses of dynamical systems through computation and concrete examples, both low-dimensional and high-dimensional, the latter arising in climate science and machine learning. 
\subsection*{Part 1 - Foundations}

\begin{itemize}
	\item One-dimensional dynamics: fixed points, periodic orbits, bifurcations, chaos
	\item Stability of fixed points and periodic orbits
	\item Hyperbolicity and invariant manifolds
	\item SRB measures and ergodicity
	\item Stochastic differential equations, Ito calculus

\end{itemize}
\subsection*{Part 2: Computational methods}
\begin{itemize}
	\item Numerical methods for ODEs, PDEs and SDEs
	\item Markov chain Monte Carlo and sampling methods
	\item Lyapunov exponents and Lyapunov vectors
	\item Bayesian filtering and data assimilation
	\item Reduced order modeling
\end{itemize}

\subsection*{Part 3: Applications}
\begin{itemize}
	\item Climate model hierarchy
	\item Optimization algorithms
	\item Transformers and sequence models
\end{itemize}

\subsection*{Part 4: Advanced topics/research frontiers}
\begin{itemize}
	\item Mori-Zwanzig formalism
	\item Data-driven approaches to transfer/Koopman operators
\end{itemize}

\section{Grading information and late policy}

This is a seminar-style course with no exams. The grade will be determined by a final project (30\%) and homeworks (70\%). There will be 7 homeworks, 6 of which are mandatory and will determine the grade. When all 7 homeworks are submitted, the lowest grade will not count toward the grade. The final project will be a research project on a topic of your choice, related to the course material. I will assist you in the selection of a project and designing its scope, if needed. The final project will be due on the last day of class. The breakdown of the final grade is as follows:
\begin{itemize}
	\item Final project: 30\%
	\item Homework: 70\% 
\end{itemize}

\textbf{Final project}: The final project has to be done individually, and the deliverables include a proposal, code, accompanying report and a 10-minute in-class presentation. A final project rubric and a set of guidelines will be posted on canvas before the proposal due date. All written material should be typed up and submitted on Gradescope.\\

\textbf{Homeworks}: there will be 7 homework assignments (due dates TBD, but spread out evenly through the semester before the final project) that will be theoretical, and often require numerical solutions. You are welcome to discuss with other students and use online resources, including AI assistants such as ChatGPT and Github CoPilot, to solve the questions. After that, however, all the submitted work should be your own. Please submit typed up homework solutions (handwritten solutions are often illegible and will not be graded) on Gradescope as a pdf. \\

\textbf{Late policy}: there is a late penalty of 25\% for a submission late by up to 24 hours, 50\% for a submission delayed beyond 24 hours and up to 48 hours. 

\section{Honor code}
\label{sec:honor}

Georgia Tech aims to cultivate a community based on trust, academic integrity, and honor. Students are expected to act according to the highest ethical standards.  For information on Georgia Tech's Academic Honor Code, please visit \href{http://www.catalog.gatech.edu/policies/honor-code/}{this link} or \href{http://www.catalog.gatech.edu/rules/18/}{this one}. 

Any student suspected of cheating or plagiarizing on a quiz, exam, or assignment will be reported to the Office of Student Integrity, who will investigate the incident and identify the appropriate penalty for violations.

Ultimately, learning and engaging with the material, and having fun in the process is most important! Assessments and homeworks are no more than good motivators to keep you accountable in the learning process. It serves no purpose to violate the honor code. 

\section{Accommodations for Students with Disabilities} 

If you are a student with learning needs that require special accommodation, contact the Office of Disability Services at (404)894-2563 or \href{http://disabilityservices.gatech.edu/}{through their website}, as soon as possible, to make an appointment to discuss your special needs and to obtain an accommodations letter.  Please also e-mail me as soon as possible in order to set up a time to discuss your learning needs. 

\section{Student-Faculty Expectations Agreement} 

At Georgia Tech we believe that it is important to strive for an atmosphere of mutual respect, acknowledgement, and responsibility between faculty members and the student body. See \href{http://www.catalog.gatech.edu/rules/22/}{the catalog} for an articulation of some basic expectation that you can have of me and that I have of you. In the end, simple respect for knowledge, hard work, and cordial interactions will help build the environment we seek. Therefore, I encourage you to remain committed to the ideals of Georgia Tech while in this class. 
%\bibliographystyle{abbrv}           % if you need a bibliography
%\bibliography{mybib}                % assuming yours is named mybib.bib


%%%%%%%%%%% end of doc
\end{document}
