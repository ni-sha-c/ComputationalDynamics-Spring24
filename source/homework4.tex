\documentclass[12pt]{article}

\usepackage{amsmath,amssymb,amsfonts,mathtools}
\usepackage[tagged, highstructure]{accessibility}
\usepackage[english]{babel}
\usepackage[utf8x]{inputenc}
\usepackage[T1]{fontenc}
\usepackage[margin=1in]{geometry}
\usepackage{scribe}
\usepackage{listings}
\usepackage{natbib,verbatim}

\usepackage{hyperref}
\hypersetup{
    colorlinks=true,
    linkcolor=blue,
    filecolor=magenta,      
    urlcolor=magenta,
    pdftitle={Homework 4},
    pdfauthor={Nisha Chandramoorthy},
    pdflang={en-US}
}

%\Scribe{Your Name}
\title{Homework 4_8803}
\LectureNumber{CSE 8803 CDS}
\LectureDate{Due Mar 20,'24 (11:59 pm ET) on Gradescope} 
\Lecturer{Cite any sources and collaborators; do not copy. See syllabus for policy.}
\LectureTitle{Homework 4}

\lstset{style=mystyle}

\begin{document}
\MakeScribeTop


In this homework, we will explore computing Lyapunov vectors and exponents. We will consider the three-variable Lorenz '63 system, which was introduced as a reduced order model for atmospheric convection. Let $x = [a,b,c]$ be a phase point, and $a(x)$ is the first component/coordinate at $x.$ The system is given by the ODEs:
\begin{align}
	\dfrac{d\varphi^t(x)}{dt} &= v(\varphi^t(x)) = \begin{bmatrix}
	\sigma(b - a) \\
	a(\rho - c) - b \\
	ab  - \beta c
	\end{bmatrix} \circ \varphi^t(x).
\end{align}
Fix $\sigma = 10, \beta = 8/3$ and $\rho = 28$, which are values at which the system is known to be chaotic. 
\begin{enumerate}
	\item[I] (2 points) Plot the Lorenz attractor. Hint: time integrate the equations starting with a random initial condition and plot the result
	\item[II] (5 points) Compute the three Lyapunov exponents. Write a code snippet, explaining each line.

	\item[III] (5 points) Are the following statements true or false for the Lorenz '63 system? Provide justification.
		\begin{enumerate}
			\item The adjoint covariant Lyapounov vector is the same as the covariant Lyapunov vector for the top LE
			\item The top (backward) Lyapunov vector is always covariant
			\item There is a zero Lyapunov exponent for the ODE system
			\item The stable adjoint Lyapunov vector is perpendicular to the unstable Lyapunov vector 
			\item There is a dense set of points on the attractor that result in different LEs than the ones computed
		\end{enumerate}
	\item[IV] (10 points) Compute and plot the unstable adjoint Lyapunov vector. Provide a code snippet/algorithm.

\end{enumerate}
\end{document}
