\documentclass[final]{beamer}
\usepackage{amsmath,amssymb,amsthm,amsfonts,graphicx}
\usepackage{eulervm,verbatim}          
\usepackage[scaled]{helvet}
\usepackage[most]{tcolorbox}
\setbeamercolor{frametitle}{fg=black,bg=white} % Colors of the block titles
\setbeamertemplate{caption}{\raggedright\insertcaption\par}
\setbeamertemplate{caption}{\raggedright\insertcaption\par}
\definecolor{darkcerulean}{rgb}{0.03, 0.27, 0.49}
\newcommand{\citesmall}[1]{[{\color{darkcerulean}\begin{small} \textbf{#1} \end{small}}]}
\setbeamertemplate{footline}[frame number]
\DeclareMathOperator*{\argmin}{arg\,min}
\usepackage{graphicx}  % Required for including images
\usepackage{bbm}
\usepackage{booktabs} % Top and bottom rules for tables
\definecolor{burgundy}{rgb}{0.5, 0.0, 0.13}
\newcommand{\highlight}[1]{{\color{burgundy} \textbf{#1}}}
\usepackage{hyperref}
\hypersetup{
    colorlinks=true,
    linkcolor=blue,
    filecolor=magenta,      
    urlcolor=magenta,
    pdftitle={CSE8803-CDS-Lecture 1},
    pdfauthor={Nisha Chandramoorthy},
    pdflang={en-US}
}



%----------------------------------------------------------------------------------------
%	TITLE SECTION 
%----------------------------------------------------------------------------------------
\title{\begin{huge}{Lecture 1: Introduction to dynamical systems theory}\end{huge}} % Poster title


\author{Nisha Chandramoorthy} % Author(s)


%----------------------------------------------------------------------------------------

\begin{document}

\frame{\titlepage}

%----------------------------------------------------------------------------------------
%	OBJECTIVES
%----------------------------------------------------------------------------------------
\begin{frame}{What is this class about?}
	\begin{itemize}
	\item Deterministic dynamical systems, i.e., systems that evolve according to a set of rules.
	\pause 
	\item Stochastic dynamical systems, i.e., systems that evolve according to a set of rules with some randomness.
	\pause
\item Questions we will not ask:
	\pause 
	\item How to control a dynamical system?
	\item How to design/model a dynamical system?
	\end{itemize}
\end{frame}
\begin{frame}{What is this class about?}
	\begin{itemize}
		\item Questions we will ask:
		\pause
		\item What is the orbit structure -- what is the geometry of the space of all orbits?
		\pause
		\item How to predict the future of a dynamical system along with quantifying uncertainties in our prediction?
		\pause
		\item How stable are the orbits?
		\pause
		\item How to represent the system using a simpler/fewer equations?
	\end{itemize}
\end{frame}

	\begin{frame}{Learning outcomes}
	\begin{itemize}
		\item To be able to do research in dynamical systems theory.
		\item ``Take the dynamical systems approach'' to a computational math problem.
		\pause
		\item Understand problems at the interface of dynamical systems and machine learning.
		\pause 
	\end{itemize}

\end{frame}
	\begin{frame}{What are the mathematical foundations?}
		\begin{itemize}
			\item Deterministic dynamics: local stability and global stability (perturbation theory of linear operators) uses real, complex and functional analysis, linear algebra and differential geometry
			\pause
		\item Ergodic theory uses measure theory (which is probability theory + analysis)
		\pause
	\item Our brief study of stochastic systems uses stochastic analysis
		\end{itemize}
	\end{frame}
	\begin{frame}{Course resources}
		\begin{itemize}
			\item 6 homeworks (65\% of grade), 1 final project (25\% of grade) and scribing (10\% of grade).
			\pause
			\item Homeworks will be a mix of theory and computation.
			\pause
			\item Final project will be related to your research interests and involve computational dynamical systems.
			\pause
		\item Syllabus, homeworks, and lecture notes will be posted on \href{https://canvas.gatech.edu}{Canvas} and \href{https://github.com/ni-sha-c/ComputationalDynamics-Spring24}{Github}.
			\pause
			\item See the syllabus section of canvas for the timeline.
			\pause
			\item See syllabus.pdf for details and topics.
			\pause
			\item Office hours: Friday 10-11 am, CODA S1323.
		\end{itemize}
	\end{frame}

	\begin{frame}{Continuous time dynamical systems or flows}
	\begin{itemize}
		\item Given an initial point $x_0$ in phase space, $\varphi^t(x_0)$ is the point in phase space at time $t$, the \emph{state} at time $t.$
		\pause
		\item Flow: $\varphi^t$, which satisfies group (composition) action.
		\pause
		\item $\varphi^0(x_0) = {\rm Id}(x_0) = x_0$.
		\pause
		\item But what is a phase space?
		\pause
		\item For us, it will be a compact manifold, denoted $M$, for deterministic systems.
		\pause 
		\item $M$ is a $d$-dimensional smooth manifold. This means that $M$ can be mapped locally to a Euclidean space with a smooth transformation.
	\end{itemize}
	\end{frame}


	\begin{frame}{Flows of vector fields}
		\begin{itemize}
			\item A vector field $v$ is a map from the phase space to the tangent space of the phase space.
			\pause
			\item $v(x) \in T_xM$.
			\pause
			\item ODE: $\dot{x} = v(x)$.
			\pause
		\item $\dfrac{d\varphi^t}{dt}(x) = v(\varphi^t(x))$.
		\pause
		\item Vector fields can also be interpreted as linear functionals on the space of smooth functions on the manifold.
		\pause
		\item $v(f) = \langle v, df \rangle$, where $df$ is the differential of $f \in \mathcal{C}^\infty(M),$ $\langle \cdot, \cdot\rangle$ is an inner product on $TM$.

		\end{itemize}
	\end{frame}
	

	\begin{frame}{Discrete-time dynamical systems or maps}
		\begin{itemize}
			\item Given an initial point $x_0$ in phase space, $F^t(x_0)$ is the point in phase space after $t$ iterations.
			\pause
			\item Map: $F$, which satisfies group (composition) action.
			\pause
			\item $F^0(x) = {\rm Id}(x) = x$.
			\pause
		\item Can be obtained from a flow by taking the time-$\delta t$ map, e.g., from time-integration of the ODE $F(x) = \varphi^{\delta t}(x)$.
			
			
		\end{itemize}
	\end{frame}

	\begin{frame}{What is the Phase space here?}
		\begin{itemize}
		\item A tube with a flame inside -- model of jet engine -- modeled using the Navier-Stokes equation.
		\pause
		\item The Lorenz system -- model of convection in the atmosphere.
		\pause
		\item Price of a stock:
			$$dS/S = \mu dt + \sigma dW$$.
			\pause
			\item $S$ is the price of the stock, $\mu$ is the drift, $\sigma$ is the volatility, $W$ is the Wiener process.
			\pause
			\item $S$ is a stochastic process, dealt with later in the course.
		\item Delay differential equations, Partial differential equations, etc. also have infinite dimensional phase spaces.

		\end{itemize}
	\end{frame}

	\begin{frame}{Analysis and basic topology}
		\begin{itemize}
			\item Topology, open and closed sets, limits.
			\pause
			\item continuity, differentiability
			\pause
			\item Compactness
			\pause
			\item Read Chapters 1-5, 7, 9 and 11 of Rudin's Principles of Mathematical Analysis.
		\end{itemize}
	\end{frame}
	\end{document}
